\subsection{Player}

All these are performed as the user 'u1' unless specified otherwise.


\subsubsection{Transfer Balance}

Here, the player wishes to transfer 1000 credits to another player. The first command shows their current balance of 10000 credits.

Next, they call \verb|TransferBalance| to transfer 1000 credits to u2. This informs them that their new balance is 9000 credits.

\begin{lstlisting}[language=SQL]
postgres=> select * from Member;
 username | balance 
----------+---------
 u1       |   10000
(1 row)

postgres=> call TransferBalance('u2', 1000);
NOTICE:  u1 => u2
NOTICE:  New Balance: 9000
CALL
postgres=> select * from Member;
 username | balance 
----------+---------
 u1       |    9000
(1 row)
\end{lstlisting}

Running the following as 'manager' shows us the transfer has been successful.

\begin{lstlisting}[language=SQL]
postgres=> select * from Member;
 username | balance 
----------+---------
 u3       |   10000
 u1       |    9000
 u2       |   11000
(3 rows)
\end{lstlisting}

If the user has insufficient funds, for example u1 wanting to transfer 1000000 to u2:

\begin{lstlisting}[language=SQL]
postgres=> call TransferBalance('u2', 1000000);
NOTICE:  u1 => u2
NOTICE:  Cannot transfer more money than you have
\end{lstlisting}

\subsubsection{View inventory}

To view the inventory, the user simply executes:

\begin{lstlisting}[language=SQL]
postgres=> select * from inventory();
   name    | price 
-----------+-------
 Potion    |     5
 Potion    |     5
 Spellbook |    40
(3 rows)
\end{lstlisting}

\subsubsection{Apply for Purchase}

Here, the user wants to purchase an item.

\begin{lstlisting}[language=SQL]
postgres=> select * from item;
 id | price |    name    
----+-------+------------
  1 |    50 | Sword
  2 |     5 | Potion
  3 |    10 | Magic Tome
  4 |    40 | Spellbook
  5 |    20 | Wand
  6 |    25 | Armour
(6 rows)
\end{lstlisting}

They, for instance, want to buy a magic tome.

\begin{lstlisting}[language=SQL]
postgres=> call ApplyToBuy(3);
CALL
\end{lstlisting}

They can then list any pending applications.

\begin{lstlisting}[language=SQL]
postgres=> select * from ListPurchaseApplications();
 id | username |  itemname  | itemprice | approved 
----+----------+------------+-----------+----------
  4 | u1       | Magic Tome |        10 | f
(1 row)
\end{lstlisting}

\subsubsection{Create Team}

Players can create teams as follows:

\begin{lstlisting}[language=SQL]
postgres=> call CreateTeam('u1_team');
NOTICE:  Create team u1_team. Leader u1
CALL
\end{lstlisting}

\subsubsection{Apply to Join Team}

Here, u2 wishes to join the previously created team 'u1\_team' (run as u2):

\begin{lstlisting}[language=SQL]
postgres=> call RequestJoinTeam('u1_team');
NOTICE:  Request to join team u1_team.
CALL
\end{lstlisting}

\subsubsection{Accept Team Join Request}

U1 can see all pending requests for his teams:

\begin{lstlisting}[language=SQL]
postgres=> select * from ListTeamRequests();
 id | teamname | requestinguser 
----+----------+----------------
  2 | u1_team  | u2
(1 row)
\end{lstlisting}


They can then accept the request:

\begin{lstlisting}[language=SQL]
postgres=> call AcceptTeamRequest(2);
CALL
\end{lstlisting}

And observe there are no further pending requests:

\begin{lstlisting}[language=SQL]
postgres=> select * from ListTeamRequests();
 id | teamname | requestinguser 
----+----------+----------------
(0 rows)
\end{lstlisting}

\subsection{Manager}

\subsubsection{Approve Purchase}
