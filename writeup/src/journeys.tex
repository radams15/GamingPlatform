\subsection{Player}

All these are performed as the user 'u1' unless specified otherwise.


\subsubsection{Transfer Balance}

Here, the player wishes to transfer 1000 credits to another player. The first command shows their current balance of 10000 credits.

Next, they call \verb|TransferBalance| to transfer 1000 credits to u2. This informs them that their new balance is 9000 credits.

\begin{lstlisting}[language=SQL]
postgres=> select * from Member;
 username | balance 
----------+---------
 u1       |   10000
(1 row)

postgres=> call TransferBalance('u2', 1000);
NOTICE:  u1 => u2
NOTICE:  New Balance: 9000
CALL
postgres=> select * from Member;
 username | balance 
----------+---------
 u1       |    9000
(1 row)
\end{lstlisting}

Running the following as 'manager' shows us the transfer has been successful.

\begin{lstlisting}[language=SQL]
postgres=> select * from Member;
 username | balance 
----------+---------
 u3       |   10000
 u1       |    9000
 u2       |   11000
(3 rows)
\end{lstlisting}

If the user has insufficient funds, for example u1 wanting to transfer 1000000 to u2:

\begin{lstlisting}[language=SQL]
postgres=> call TransferBalance('u2', 1000000);
NOTICE:  u1 => u2
NOTICE:  Cannot transfer more money than you have
\end{lstlisting}

\subsubsection{View inventory}

To view the inventory, the user simply executes:

\begin{lstlisting}[language=SQL]
postgres=> select * from inventory();
   name    | price 
-----------+-------
 Potion    |     5
 Potion    |     5
 Spellbook |    40
(3 rows)
\end{lstlisting}

\subsubsection{Apply for Purchase}

Here, the user wants to purchase an item.

\begin{lstlisting}[language=SQL]
postgres=> select * from item;
 id | price |    name    
----+-------+------------
  1 |    50 | Sword
  2 |     5 | Potion
  3 |    10 | Magic Tome
  4 |    40 | Spellbook
  5 |    20 | Wand
  6 |    25 | Armour
(6 rows)
\end{lstlisting}

They, for instance, want to buy a magic tome.

\begin{lstlisting}[language=SQL]
postgres=> call ApplyToBuy(3);
CALL
\end{lstlisting}

They can then list any pending applications.

\begin{lstlisting}[language=SQL]
postgres=> select * from ListPurchaseApplications();
 id | username |  itemname  | itemprice | approved 
----+----------+------------+-----------+----------
  4 | u1       | Magic Tome |        10 | f
(1 row)
\end{lstlisting}

\subsubsection{Create Team}

Players can create teams as follows:

\begin{lstlisting}[language=SQL]
postgres=> call CreateTeam('u1_team');
NOTICE:  Create team u1_team. Leader u1
CALL
\end{lstlisting}

\subsubsection{Apply to Join Team}

Here, u2 wishes to join the previously created team 'u1\_team' (run as u2):

\begin{lstlisting}[language=SQL]
postgres=> call RequestJoinTeam('u1_team');
NOTICE:  Request to join team u1_team.
CALL
\end{lstlisting}

\subsubsection{Accept Team Join Request}

U1 can see all pending requests for his teams:

\begin{lstlisting}[language=SQL]
postgres=> select * from ListTeamRequests();
 id | teamname | requestinguser 
----+----------+----------------
  2 | u1_team  | u2
(1 row)
\end{lstlisting}


They can then accept the request:

\begin{lstlisting}[language=SQL]
postgres=> call AcceptTeamRequest(2);
CALL
\end{lstlisting}

And observe there are no further pending requests:

\begin{lstlisting}[language=SQL]
postgres=> select * from ListTeamRequests();
 id | teamname | requestinguser 
----+----------+----------------
(0 rows)
\end{lstlisting}

\subsection{Manager}

All these are performed as the manager user 'm1' unless specified otherwise.

\subsubsection{Approve Purchase}

Managers can list pending purchases:

\begin{lstlisting}[language=SQL]
postgres=> select * from ListPurchaseApplications();
 id | username |  itemname  | itemprice | approved 
----+----------+------------+-----------+----------
  4 | u1       | Magic Tome |        10 | f
(1 row)
\end{lstlisting}

Approving the purchase shows the balance of u1 being decremented by the price of the item.

Similarly to transfers, if the purchaser does not have enough funds to complete the transaction, this step will fail.

The checking for funds is done here rather than at the request stage to ensure the user does not decrease their balance between
the times of the request and approval.

\begin{lstlisting}[language=SQL]
postgres=> call ApprovePurchaseApplication(4);
NOTICE:  New balance: 8990, price 10
CALL
\end{lstlisting}

Now there are no further pending applications.

\begin{lstlisting}[language=SQL]
postgres=> select * from ListPurchaseApplications();
 id | username | itemname | itemprice | approved 
----+----------+----------+-----------+----------
(0 rows)
\end{lstlisting}

Running the following as u1 we can see the magic tome has been added.

\begin{lstlisting}[language=SQL]
postgres=> select * from Inventory();
    name    | price 
------------+-------
 Potion     |     5
 Potion     |     5
 Magic Tome |    10
 Spellbook  |    40
(4 rows)
\end{lstlisting}


\section{Procedures and Functions}

The following functions and procedures are defined:

\begin{itemize}
    \item \verb|FUNCTION Inventory()| - List all purchased items in the caller's inventory.
    \item \verb|PROCEDURE ApplyToBuy(itemId INTEGER)| - Create an application to purchase an item by creating an entry in the \verb|MemberPurchase| table.
    \item \verb|PROCEDURE ApprovePurchaseApplication(applicationId INTEGER)| - Manager command to approve a transaction request.
        \begin{itemize}
            \item First checks if application is already approved.
            \item Then checks if the user has enough credits to buy the item.
            \item Update the user's balance.
            \item Set the application approved boolean to true.
            \item Add the item to the user's inventory.
        \end{itemize}
    \item \verb|FUNCTION ListPurchaseApplications()| - List all transaction requests, either just for the current user (player), or for all players (manager).
        \begin{itemize}
            \item Select all items from \verb|MemberPurchase| and join the item table to get the item names.
        \end{itemize}
    \item \verb|PROCEDURE TransferBalance(toUser TEXT, amount INTEGER)| - Transfer 'amount' credits to 'toUser'. This does not require authorisation.
        \begin{itemize}
            \item First checks if the user is transferring to themselves.
            \item Then checks if the calling user has enough credits to transfer the requested amount.
            \item Then calculate the new balances and updates the user's balances.
        \end{itemize}
    \item \verb|PROCEDURE CreateTeam(name TEXT)| - Creates a new team with specified name, sets caller as the team leader.
        \begin{itemize}
            \item Creates a new value in \verb|Team|.
            \item Then adds an entry to \verb|TeamMember| for the creator.
        \end{itemize}
    \item \verb|PROCEDURE RequestJoinTeam(team TEXT)| - Creates a team join application for specified team.
        \begin{itemize}
            \item First checks if a request has already been sent.
            \item Next adds an entry to \verb|TeamJoinRequest| with the current user and the requested team.
        \end{itemize}
    \item \verb|FUNCTION ListTeamRequests()| - Lists requests to join teams the caller is a leader of from the \verb|TeamJoinRequest| table.
    \item \verb|PROCEDURE AcceptTeamRequest(requestId INTEGER)| - Accept the specified team join request.
        \begin{itemize}
            \item First checks if the caller actually owns the team (is listed in the leader field).
            \item Then sets the approved flag on the \verb|TeamJoinRequest| table.
            \item Then adds the new member into the \verb|TeamMember| table.
        \end{itemize}
\end{itemize}

The reasoning for not utilising views for some of the functions above is that the views bypass the row-level security. If PostgreSQL was running at version 16 or higher then the attribute \verb|SECURITY INVOKER| could have been used, but this is unsupported on the specified PostgreSQL version 13.
